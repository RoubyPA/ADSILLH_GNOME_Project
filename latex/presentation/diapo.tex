\documentclass{beamer}

\usepackage[utf8]{inputenc}
\usepackage{graphicx}
\usetheme{Berkeley} 
\usecolortheme{default}
\setbeamertemplate{footline}[frame number]

%% Title frame info
\title{Licence Pro ADSILLH}
\subtitle{GNOME-Games / GNOME-Music}
%% \institute{ADSILLH}
\author{Pierre-Antoine Rouby\\Gautier Delacour\\
  David Tabarie\\Kevin Carsoule\\Florian Darfeuille}
\date{Année 2017/2018}
\logo{\includegraphics[width=1.5cm]{../images/logo_univ.jpg}}

% ce code permet d'afficher le sommaire à chaque section,
% en mettant en valeur la section courante (utile ?)
% \AtBeginSection[]
% {
%   \begin{frame}
%     \frametitle{Sommaire}
%     \tableofcontents[currentsection]
%   \end{frame}
% }

\begin{document}
\frame{\titlepage}

\begin{frame}
  \frametitle{Sommaire}
  \tableofcontents
\end{frame}

\section{Le projet GNOME}
\begin{frame}
  \frametitle{Le projet GNOME}
  \includegraphics[scale=0.2]{images/GnomeScreen.png}
\end{frame}

\subsection{Présentation}
\begin{frame}
  \frametitle{Présentation}
  \begin{itemize}
  \item Environnement de bureaux \pause
    \begin{itemize}
    \item Gestionnaire de fenêtres (Window Manager) \pause
    \item Nombreux logiciels \pause
      \begin{itemize} % CHECK
      \item Lecteur vidéo
      \item Visionneur d'image
      \item Navigateur de fichier
      \item etc. \pause
      \end{itemize}
    \end{itemize}
  \item Utilisé dans de nombres système Linux \pause
    \begin{itemize}
    \item Debian (Ubuntu)
    \item Fedora (Red Hat)
    \item Gentoo \pause
    \end{itemize}
  \item Actuellement en version 3
  \end{itemize}
\end{frame}

\subsection{Les outils collaboratifs}
\begin{frame}
  \frametitle{Les outils collaboratifs}
  
  %% Solution 1
  \begin{columns}
    \column{0.5\textwidth}
    \begin{itemize}
    \item Bugzilla
    \item Wiki GNOME
    \item Git
    \item irc
    \end{itemize}
    \column{0.5\textwidth}
    \begin{figure}
      \includegraphics[scale=0.1]{images/bugzilla-logo.png} \\
      \includegraphics[scale=0.04]{images/gnome-logo.png} \\
      \includegraphics[scale=0.03]{images/git-logo.png} \\
      \includegraphics[scale=0.07]{images/irc-logo.png}
    \end{figure}
  \end{columns}

  %% %% Solution 2
  %% \begin{itemize}
  %% \item Bugzilla
  %% \item Wiki GNOME
  %% \item Git
  %% \item irc
  %% \end{itemize}
  %% \begin{figure}
  %%   \includegraphics[scale=0.1]{images/bugzilla-logo.png} \hspace{1cm}
  %%   \includegraphics[scale=0.04]{images/gnome-logo.png} \hspace{1cm}
  %%   \includegraphics[scale=0.03]{images/git-logo.png} \hspace{1cm}
  %%   \includegraphics[scale=0.07]{images/irc-logo.png}
  %% \end{figure}

  %% %% Solution 3
  %% \begin{itemize}
  %% \item Bugzilla
  %%   \includegraphics[scale=0.1]{images/bugzilla-logo.png}
  %% \item GnomeWiki
  %%   \includegraphics[scale=0.04]{images/gnome-logo.png}
  %% \item Git
  %%   \includegraphics[scale=0.03]{images/git-logo.png}
  %% \item IRC
  %%   \includegraphics[scale=0.07]{images/irc-logo.png}
  %% \end{itemize}
\end{frame}

\subsection{Newcomers}
\begin{frame}
  \frametitle{Newcomers}
  \begin{itemize}
  \item Outils pour les nouveaux contributeurs \pause
    \begin{itemize}
    \item Builder
    \item Liste de projets
    \item Liste de bug simples \pause
    \end{itemize}
  \item Canal de communication dédié \pause
    \begin{itemize}
    \item Channel irc (\#newcomers)
    \item Mailing list
    \end{itemize}
  \end{itemize}
\end{frame}

\section{GNOME-Games}
\begin{frame}
  \frametitle{GNOME-Games}
  \includegraphics[scale=0.25]{images/screen-games.png}
\end{frame}

\subsection{Présentation}
\begin{frame}
  \frametitle{Présentation}
  \begin{itemize}
  \item Collection de jeux \pause
    \begin{itemize}
    \item Jeux retro
    \item Natif Linux \pause
    \end{itemize}
  \item Basé sur plusieurs bibliothèques \pause
    \begin{itemize}
    \item libmanette
    \item retro-gtk
    \end{itemize}
  \end{itemize}
\end{frame}

\subsection{Contributions}
\begin{frame}
  \frametitle{Contributions à Games}
  \begin{itemize}
  \item Traçage d'un bug PulseAudio \pause
    \begin{itemize}
      \item Problème complexe en apparence \pause
    \end{itemize}
  \item Patch : Modifications du fichier 'HACKING' \pause
    \begin{itemize}
    \item Utilisation de la syntaxe markdown
    \item Modifications demandé par le mainteneur
    \item Accepté \pause
    \end{itemize}
  \item Patch : Modifications du fichier 'README.md' \pause
    \begin{itemize}
    \item Contribution à ``retro-gtk''
    \item Utilisation de la syntaxe markdown
    \item Pas encore accepté \pause
    \end{itemize}
  \item Patch : Ajout d'un event 'retour au menu'
  \end{itemize}
\end{frame}

\subsection{Conclusion}
\begin{frame}
  \frametitle{Conclusion Games}
  \begin{itemize}
  \item Les outils collaboratifs \pause
    \begin{itemize}
    \item Git \pause
    \item irc \pause
    \item Tracker de bug \pause
    \end{itemize}
  \item Les prochaines contributions
  \end{itemize}
\end{frame}

\section{GNOME-Music}
\begin{frame}
  \frametitle{Le projet Gnome Music}
  \includegraphics[scale=0.325]{images/gnome-music-app.png}
\end{frame}

\subsection{Contributions effectués sur GNOME Music}
\begin{frame}
  \frametitle{Contributions effectuées}
  \begin{itemize}
  \item Rapport de bug sur le Builder
  \item Patch Pep8
  \item Patch Bouton retour arrière
  \end{itemize}
\end{frame}

\subsection{Conclusion Music}
\begin{frame}
  \frametitle{Conclusion Music}
  \begin{itemize}
  \item Le travail collaboratif
  \item La suite
  \end{itemize}
\end{frame}

\section{Question}
\begin{frame}
  \frametitle{Question ?}
  %% Merci ?
\end{frame}

\end{document}
